%%%%%%%%%%%%%%%%%%%%%%%%%%%%%%%%%
% PACKAGE IMPORTS
%%%%%%%%%%%%%%%%%%%%%%%%%%%%%%%%%

\usepackage[tmargin=2cm,rmargin=1in,lmargin=1in,margin=0.85in,bmargin=2cm,footskip=.2in]{geometry}
\usepackage{amsmath,amsfonts,amsthm,amssymb,mathtools}
\usepackage{bookmark}
\usepackage{enumitem}
\usepackage{hyperref,theoremref}
\hypersetup{
    pdftitle={},
    colorlinks=true, linkcolor=doc!90,
    bookmarksnumbered=true,
    bookmarksopen=true
}
\usepackage[most,many,breakable]{tcolorbox}
\usepackage{xcolor}
\usepackage{graphicx}
%\graphicspath{ {./images/} } 
\usepackage{varwidth}
\usepackage{authblk}
\usepackage{nameref}
\usepackage{multicol,array}
\usepackage{tikz-cd}
\usepackage{cancel}
\usepackage{pgfplots}
\usepackage{graphics}
\usepackage{quotchap}
\usepackage[compact]{titlesec}% http://ctan.org/pkg/titlesec
\titleformat{\section}
{\normalfont\Large\bfseries}{\thesection}{0pt}{\hfill}[{\titlerule[1.8pt]}]
\titlespacing*{\section}
{0em} %left
{0em} %before
{3em} %after/below

\titleformat{\subsection}
{\normalfont\Large\bfseries}{\thesubsection}{0pt}{\hfill}[{\titlerule[0.8pt]}]
\titlespacing*{\subsection}
{0em} %left
{0em} %before
{3em} %after/below

%%%%%%%%%%%%%%%%%%%%%%%%%%%%%%
% SELF MADE COLORS
%%%%%%%%%%%%%%%%%%%%%%%%%%%%%%

\usetikzlibrary{ shapes.geometric }
\usetikzlibrary{calc}
\usepackage{anyfontsize}

\definecolor{myg}{RGB}{56, 140, 70}
\definecolor{myb}{RGB}{45, 111, 177}
\definecolor{myr}{RGB}{199, 68, 64}
\definecolor{mytheorembg}{HTML}{F2F2F9}
\definecolor{mytheoremfr}{HTML}{00007B}
\definecolor{myexamplebg}{HTML}{F2FBF8}
\definecolor{myexamplefr}{HTML}{88D6D1}
\definecolor{myexampleti}{HTML}{2A7F7F}
\definecolor{mydefinitbg}{HTML}{E5E5FF}
\definecolor{mydefinitfr}{HTML}{3F3FA3}
\definecolor{notesgreen}{RGB}{0,162,0}
\definecolor{myp}{RGB}{197, 92, 212}
\definecolor{mygr}{HTML}{2C3338}
\definecolor{myred}{RGB}{127,0,0}
\definecolor{myyellow}{RGB}{169,121,69}
\definecolor{OrangeRed}{HTML}{ED135A}
\definecolor{Dandelion}{HTML}{FDBC42}
\definecolor{light-gray}{gray}{0.95}
\definecolor{Emerald}{HTML}{00A99D}
\definecolor{RoyalBlue}{HTML}{0071BC}


%%%%%%%%%%%%%%%%%%%%%%%%%%%%
% TCOLORBOX SETUPS
%%%%%%%%%%%%%%%%%%%%%%%%%%%%
% Theorem
\definecolor{theorembg}{RGB}{230, 230, 255} % Light blue background for Theorem
\definecolor{theoremfr}{RGB}{0, 0, 139} % Dark blue border for Theorem

% Corollary
\definecolor{corollarybg}{RGB}{255, 250, 240} % Light cream background for Corollary
\definecolor{corollaryfr}{RGB}{255, 69, 0} % Red border for Corollary

% Lemma
\definecolor{lemmabg}{RGB}{255, 255, 224} % Light yellow background for Lemma
\definecolor{lemmafr}{RGB}{255, 215, 0} % Dark yellow border for Lemma

% Claim
\definecolor{claimbg}{RGB}{240, 255, 240} % Light mint background for Claim
\definecolor{claimfr}{RGB}{0, 100, 0} % Dark green border for Claim

% Weak Conjecture
\definecolor{wconcernbg}{RGB}{240, 248, 255} % Light azure background for Weak Conjecture
\definecolor{wconcernfr}{RGB}{0, 0, 139} % Dark blue border for Weak Conjecture

% Example
\definecolor{examplebg}{RGB}{255, 240, 245} % Light pink background for Example
\definecolor{examplefr}{RGB}{255, 20, 147} % Dark pink border for Example

% Definition
\definecolor{definitionbg}{RGB}{255, 250, 240} % Light salmon background for Definition
\definecolor{definitionfr}{RGB}{255, 69, 0} % Red-orange border for Definition

% Open Problem
\definecolor{openbg}{RGB}{240, 255, 240} % Light honeydew background for Open Problem
\definecolor{openfr}{RGB}{0, 128, 0} % Dark green border for Open Problem

% Question
\definecolor{questionbg}{RGB}{255, 248, 220} % Light beige background for Question
\definecolor{questionfr}{RGB}{255, 215, 0} % Gold border for Question

% Note
\definecolor{notebg}{RGB}{255, 255, 224} % Light khaki background for Note
\definecolor{notefr}{RGB}{189, 183, 107} % Dark khaki border for Note

\tcbuselibrary{theorems,skins,hooks}

% Theorem Environment
\newtcbtheorem[number within=section]{Theorem}{Theorem}
{%
    enhanced,
    breakable,
    colback = theorembg,
    frame hidden,
    boxrule = 0sp,
    borderline west = {2pt}{0pt}{theoremfr},
    sharp corners,
    detach title,
    before upper = \tcbtitle\par\smallskip,
    coltitle = theoremfr,
    fonttitle = \bfseries\sffamily,
    description font = \mdseries,
    separator sign none,
    segmentation style={solid, theoremfr},
}
{th}

% Corollary Environment
\newtcbtheorem[number within=section]{Corollary}{Corollary}
{%
    enhanced,
    breakable,
    colback = corollarybg,
    frame hidden,
    boxrule = 0sp,
    borderline west = {2pt}{0pt}{corollaryfr},
    sharp corners,
    detach title,
    before upper = \tcbtitle\par\smallskip,
    coltitle = corollaryfr,
    fonttitle = \bfseries\sffamily,
    description font = \mdseries,
    separator sign none,
    segmentation style={solid, corollaryfr},
}
{cor}

% Lemma Environment
\newtcbtheorem[number within=section]{Lemma}{Lemma}
{%
    enhanced,
    breakable,
    colback = lemmabg,
    frame hidden,
    boxrule = 0sp,
    borderline west = {2pt}{0pt}{lemmafr},
    sharp corners,
    detach title,
    before upper = \tcbtitle\par\smallskip,
    coltitle = lemmafr,
    fonttitle = \bfseries\sffamily,
    description font = \mdseries,
    separator sign none,
    segmentation style={solid, lemmafr},
}
{lem}

% Claim Environment
\newtcbtheorem[number within=section]{Claim}{Claim}
{%
    enhanced,
    breakable,
    colback = claimbg,
    frame hidden,
    boxrule = 0sp,
    borderline west = {2pt}{0pt}{claimfr},
    sharp corners,
    detach title,
    before upper = \tcbtitle\par\smallskip,
    coltitle = claimfr,
    fonttitle = \bfseries\sffamily,
    description font = \mdseries,
    separator sign none,
    segmentation style={solid, claimfr},
}
{clm}

% Weak Conjecture Environment
\newtcbtheorem[number within=section]{wconc}{Weak Conjecture}
{%
    enhanced,
    breakable,
    colback = wconcernbg,
    frame hidden,
    boxrule = 0sp,
    borderline west = {2pt}{0pt}{wconcernfr},
    sharp corners,
    detach title,
    before upper = \tcbtitle\par\smallskip,
    coltitle = wconcernfr,
    fonttitle = \bfseries\sffamily,
    description font = \mdseries,
    separator sign none,
    segmentation style={solid, wconcernfr},
}
{wc}

% Example Environment
\newtcbtheorem[number within=section]{Example}{Example}
{%
    enhanced,
    breakable,
    colback = examplebg,
    frame hidden,
    boxrule = 0sp,
    borderline west = {2pt}{0pt}{examplefr},
    sharp corners,
    detach title,
    before upper = \tcbtitle\par\smallskip,
    coltitle = examplefr,
    fonttitle = \bfseries\sffamily,
    description font = \mdseries,
    separator sign none,
    segmentation style={solid, examplefr},
}
{ex}

% Definition Environment
\newtcbtheorem[number within=section]{Definition}{Definition}
{%
    enhanced,
    breakable,
    colback = definitionbg,
    frame hidden,
    boxrule = 0sp,
    borderline west = {2pt}{0pt}{definitionfr},
    sharp corners,
    detach title,
    before upper = \tcbtitle\par\smallskip,
    coltitle = definitionfr,
    fonttitle = \bfseries\sffamily,
    description font = \mdseries,
    separator sign none,
    segmentation style={solid, definitionfr},
}
{dfn}

% Open Problem Environment
\newtcbtheorem[number within=section]{open}{Open Problem}
{%
    enhanced,
    breakable,
    colback = openbg,
    frame hidden,
    boxrule = 0sp,
    borderline west = {2pt}{0pt}{openfr},
    sharp corners,
    detach title,
    before upper = \tcbtitle\par\smallskip,
    coltitle = openfr,
    fonttitle = \bfseries\sffamily,
    description font = \mdseries,
    separator sign none,
    segmentation style={solid, openfr},
}
{opn}

% Question Environment
\newtcbtheorem[number within=section]{Question}{Question}
{%
    enhanced,
    breakable,
    colback = questionbg,
    frame hidden,
    boxrule = 0sp,
    borderline west = {2pt}{0pt}{questionfr},
    sharp corners,
    detach title,
    before upper = \tcbtitle\par\smallskip,
    coltitle = questionfr,
    fonttitle = \bfseries\sffamily,
    description font = \mdseries,
    separator sign none,
    segmentation style={solid, questionfr},
}
{qs}

% Note Environment

\newtcbtheorem[number within=section]{Note}{Note}
{%
    enhanced,
    breakable,
    colback = notebg,
    frame hidden,
    boxrule = 0sp,
    borderline west = {2pt}{0pt}{notefr},
    sharp corners,
    detach title,
    before upper = \tcbtitle\par\smallskip,
    coltitle = notefr,
    fonttitle = \bfseries\sffamily,
    description font = \mdseries,
    separator sign none,
    segmentation style={solid, notefr},
}
{nt}

\newcommand{\dfn}[3][]{\begin{Definition}{#2}{#1}#3\end{Definition}}
\newcommand{\thm}[3][]{\begin{Theorem}{#2}{#1}#3\end{Theorem}}
\newcommand{\cor}[3][]{\begin{Corollary}{#2}{#1}#3\end{Corollary}}
\newcommand{\lem}[3][]{\begin{Lemma}{#2}{#1}#3\end{Lemma}}
\newcommand{\clm}[3][]{\begin{Claim}{#2}{#1}#3\end{Claim}}
\newcommand{\ex}[3][]{\begin{Example}{#2}{#1}#3\end{Example}}
\newcommand{\opn}[3][]{\begin{open}{#2}{#1}#3\end{open}}
\newcommand{\qs}[3][]{\begin{Question}{#2}{#1}#3\end{Question}}
\newcommand{\nt}[3][]{\begin{Note}{#2}{#1}#3\end{Note}}
\newcommand{\wc}[3][]{\begin{wconc}{#2}{#1}#3\end{wconc}}
\newcommand{\pf}[2]{\begin{myproof}[#1]#2\end{myproof}}

\newcommand*\circled[1]{\tikz[baseline=(char.base)]{
        \node[shape=circle,draw,inner sep=1pt] (char) {#1};}}
\newcommand\getcurrentref[1]{%
    \ifnumequal{\value{#1}}{0}
    {??}
    {\the\value{#1}}%
}
\newcommand{\getCurrentSectionNumber}{\getcurrentref{section}}
\newenvironment{myproof}[1][\proofname]{%
    \proof[\bfseries #1: ]%
}{\endproof}
\newcounter{mylabelcounter}

\makeatletter
\newcommand{\setword}[2]{%
    \phantomsection
    #1\def\@currentlabel{\unexpanded{#1}}\label{#2}%
}
\makeatother

\tikzset{
    symbol/.style={
            draw=none,
            every to/.append style={
                    edge node={node [sloped, allow upside down, auto=false]{$#1$}}}
        }
}

%%%%%%%%%%%%%%%%%%%%%%%%%%%%%%%%%%%%%%%%%%%
% TABLE OF CONTENTS 
%%%%%%%%%%%%%%%%%%%%%%%%%%%%%%%%%%%%%%%%%%%
\usepackage{tikz}
\definecolor{doc}{RGB}{255,0,0}
\usepackage{titletoc}

\contentsmargin{0cm}

\titlecontents{chapter}[3.7pc]
{\addvspace{30pt}%
    \begin{tikzpicture}[remember picture, overlay]%
        \draw[fill=doc!90,draw=doc!90] (-7,-.1) rectangle (-0.7,.5);%
        \pgftext[left,x=-3.6cm,y=0.2cm]{\color{white}\Large\sc\bfseries Chapter\ \thecontentslabel};%
    \end{tikzpicture}\color{doc!90}\large\sc\bfseries}%
{}
{}
{\;\titlerule\;\large\sc\bfseries Page \thecontentspage
    \begin{tikzpicture}[remember picture, overlay]
        \draw[fill=doc!90,draw=doc!90] (2pt,0) rectangle (4,0.1pt);
    \end{tikzpicture}}%

\titlecontents{section}[3.7pc]
{\addvspace{2pt}}
{\contentslabel[\thecontentslabel]{2pc}}
{}
{\hfill\small \thecontentspage}
[]

\titlecontents{subsection}[5.7pc]  % Adjusted indent for subsections
{\addvspace{2pt}\small}
{\contentslabel[\thecontentslabel]{2pc}}
{}
{\hfill\small \thecontentspage}
[]

\makeatletter
\renewcommand{\tableofcontents}{%
    \chapter*{%
      \vspace*{-20\p@}%
      \begin{tikzpicture}[remember picture, overlay]%
          \pgftext[right,x=15cm,y=0.2cm]{\color{doc!90}\Huge\sc\bfseries \contentsname};%
          \draw[fill=doc!90,draw=doc!90] (13,-.75) rectangle (20,1);%
          \clip (13,-.75) rectangle (20,1);
          \pgftext[right,x=15cm,y=0.2cm]{\color{white}\Huge\sc\bfseries \contentsname};%
      \end{tikzpicture}}%
    \@starttoc{toc}}
\makeatother


%%%%%%%%%%%%%%%%%%%%%%%%%%%%%%%%%%%%%%%%%%%
% Title Page 
%%%%%%%%%%%%%%%%%%%%%%%%%%%%%%%%%%%%%%%%%%%
\definecolor{doc}{RGB}{0, 100, 0}
\newcommand{\mytitleb}[4]{\begin{tikzpicture}[overlay,remember picture]

        % Background color
        \fill[
            black!2]
        (current page.south west) rectangle (current page.north east);

        % Rectangles
        \shade[
            left color=doc,
            right color=doc!60,
            transform canvas ={rotate around ={45:($(current page.north west)+(0,-6)$)}}]
        ($(current page.north west)+(0,-6)$) rectangle ++(9,1.5);

        \shade[
            left color=lightgray,
            right color=lightgray!50,
            rounded corners=0.75cm,
            transform canvas ={rotate around ={45:($(current page.north west)+(.5,-10)$)}}]
        ($(current page.north west)+(0.5,-10)$) rectangle ++(15,1.5);

        \shade[
            left color=doc,
            right color=doc!60,
            rounded corners=0.4cm,
            transform canvas ={rotate around ={45:($(current page.north)+(-1.5,-3)$)}}]
        ($(current page.north)+(-1.5,-3)$) rectangle ++(9,0.8);

        \shade[
            left color=doc,
            right color=doc!60,
            rounded corners=0.9cm,
            transform canvas ={rotate around ={45:($(current page.north)+(-3,-8)$)}}] ($(current page.north)+(-3,-8)$) rectangle ++(15,1.8);

        \shade[
            left color=doc,
            right color=doc!60,
            rounded corners=0.9cm,
            transform canvas ={rotate around ={45:($(current page.north west)+(4,-15.5)$)}}]
        ($(current page.north west)+(4,-15.5)$) rectangle ++(30,1.8);

        \shade[
            left color=doc,
            right color=doc!60,
            rounded corners=0.75cm,
            transform canvas ={rotate around ={45:($(current page.north west)+(13,-10)$)}}]
        ($(current page.north west)+(13,-10)$) rectangle ++(15,1.5);

        \shade[
            left color=lightgray,
            rounded corners=0.3cm,
            transform canvas ={rotate around ={45:($(current page.north west)+(18,-8)$)}}]
        ($(current page.north west)+(18,-8)$) rectangle ++(15,0.6);

        \shade[
            left color=lightgray,
            rounded corners=0.4cm,
            transform canvas ={rotate around ={45:($(current page.north west)+(19,-5.65)$)}}]
        ($(current page.north west)+(19,-5.65)$) rectangle ++(15,0.8);

        \shade[
            left color=doc,
            right color=doc!60,
            rounded corners=0.6cm,
            transform canvas ={rotate around ={45:($(current page.north west)+(20,-9)$)}}]
        ($(current page.north west)+(20,-9)$) rectangle ++(14,1.2);

        % Year
        \draw[ultra thick,gray]
        ($(current page.center)+(5,2)$) -- ++(0,-3cm)
        node[
            midway,
            left=0.25cm,
            text width=5cm,
            align=right,
            black!75
        ]
        {
            {\fontsize{25}{30} \selectfont \bf  Lecture\\[10pt] Notes}
        }
        node[
            midway,
            right=0.25cm,
            text width=6cm,
            align=left,
            doc]
        {
            {\fontsize{72}{86.4} \selectfont #4}
        };

        % Title
        \node[align=center] at ($(current page.center)+(0,-5)$)
        {
        {\fontsize{60}{72} \selectfont {{#1}}} \\[1cm]
        {\fontsize{16}{19.2} \selectfont \textcolor{doc}{ \bf #2}}\\[3pt]
        #3};
    \end{tikzpicture}
}


%%%%%%%%%%%%%%%%%%%
%  Todo Commands  %
%%%%%%%%%%%%%%%%%%%
\usepackage{xargs}
\usepackage[colorinlistoftodos, textwidth=0.8\marginparwidth]{todonotes}

\newcommandx{\unsure}[2][1=]{\todo[size=\scriptsize, linecolor=red,backgroundcolor=red!25,bordercolor=red,#1]{#2}}
\newcommandx{\fix}[2][1=]{\todo[size=\scriptsize, linecolor=orange,backgroundcolor=orange!25,bordercolor=orange,#1]{#2}}
\newcommandx{\info}[2][1=]{\todo[size=\scriptsize, linecolor=blue,backgroundcolor=blue!25,bordercolor=blue,#1]{#2}}