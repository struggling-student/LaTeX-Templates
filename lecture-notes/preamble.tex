%%%%%%%%%%%%%%%%%%%%%%%%%%%%%%%%%
% PACKAGE IMPORTS
%%%%%%%%%%%%%%%%%%%%%%%%%%%%%%%%%

\usepackage[tmargin=2cm,rmargin=1in,lmargin=1in,margin=0.85in,bmargin=2cm,footskip=.2in]{geometry} 
\usepackage{amsmath,amsfonts,amsthm,amssymb,mathtools}
\usepackage{bookmark}
\usepackage{enumitem}
\usepackage{hyperref,theoremref}
\hypersetup{
	pdftitle={}, 
	colorlinks=true, linkcolor=doc!90,
	bookmarksnumbered=true,
	bookmarksopen=true
}
\usepackage[most,many,breakable]{tcolorbox}
\usepackage{xcolor}
\usepackage{graphicx}
%\graphicspath{ {./images/} } 
\usepackage{varwidth}
\usepackage{authblk}
\usepackage{nameref}
\usepackage{multicol,array}
\usepackage{tikz-cd}
\usepackage{cancel}
\usepackage{pgfplots}
\usepackage{lipsum}

%%%%%%%%%%%%%%%%%%%%%%%%%%%%%%
% SELF MADE COLORS
%%%%%%%%%%%%%%%%%%%%%%%%%%%%%%

\usetikzlibrary{ shapes.geometric }
\usetikzlibrary{calc}
\usepackage{anyfontsize}

\definecolor{myg}{RGB}{56, 140, 70}
\definecolor{myb}{RGB}{45, 111, 177}
\definecolor{myr}{RGB}{199, 68, 64}
\definecolor{mytheorembg}{HTML}{F2F2F9}
\definecolor{mytheoremfr}{HTML}{00007B}
\definecolor{myexamplebg}{HTML}{F2FBF8}
\definecolor{myexamplefr}{HTML}{88D6D1}
\definecolor{myexampleti}{HTML}{2A7F7F}
\definecolor{mydefinitbg}{HTML}{E5E5FF}
\definecolor{mydefinitfr}{HTML}{3F3FA3}
\definecolor{notesgreen}{RGB}{0,162,0}
\definecolor{myp}{RGB}{197, 92, 212}
\definecolor{mygr}{HTML}{2C3338}
\definecolor{myred}{RGB}{127,0,0}
\definecolor{myyellow}{RGB}{169,121,69}
\definecolor{OrangeRed}{HTML}{ED135A}
\definecolor{Dandelion}{HTML}{FDBC42}
\definecolor{light-gray}{gray}{0.95}
\definecolor{Emerald}{HTML}{00A99D}
\definecolor{RoyalBlue}{HTML}{0071BC}
% change this color to change the color of the title page and table of contents
\definecolor{doc}{RGB}{79, 134, 247}

%%%%%%%%%%%%%%%%%%%%%%%%%%%%
% TCOLORBOX SETUPS
%%%%%%%%%%%%%%%%%%%%%%%%%%%%

\setlength{\parindent}{1cm}

%================================
% THEOREM BOX
%================================

\tcbuselibrary{theorems,skins,hooks}
\newtcbtheorem[number within=section]{Theorem}{Theorem}
{%
	enhanced,
	breakable,
	colback = mytheorembg,
	frame hidden,
	boxrule = 0sp,
	borderline west = {2pt}{0pt}{mytheoremfr},
	sharp corners,
	detach title,
	before upper = \tcbtitle\par\smallskip,
	coltitle = mytheoremfr,
	fonttitle = \bfseries\sffamily,
	description font = \mdseries,
	separator sign none,
	segmentation style={solid, mytheoremfr},
}
{th}

\tcbuselibrary{theorems,skins,hooks}
\newtcbtheorem[number within=chapter]{theorem}{Theorem}
{%
	enhanced,
	breakable,
	colback = mytheorembg,
	frame hidden,
	boxrule = 0sp,
	borderline west = {2pt}{0pt}{mytheoremfr},
	sharp corners,
	detach title,
	before upper = \tcbtitle\par\smallskip,
	coltitle = mytheoremfr,
	fonttitle = \bfseries\sffamily,
	description font = \mdseries,
	separator sign none,
	segmentation style={solid, mytheoremfr},
}
{th}

\tcbuselibrary{theorems,skins,hooks}
\newtcolorbox{Theoremcon}
{%
	enhanced
	,breakable
	,colback = mytheorembg
	,frame hidden
	,boxrule = 0sp
	,borderline west = {2pt}{0pt}{mytheoremfr}
	,sharp corners
	,description font = \mdseries
	,separator sign none
}

%================================
% Corollery
%================================

\tcbuselibrary{theorems,skins,hooks}
\newtcbtheorem[number within=section]{corolary}{Corollary}
{%
	enhanced
	,breakable
	,colback = myp!10
	,frame hidden
	,boxrule = 0sp
	,borderline west = {2pt}{0pt}{myp!85!black}
	,sharp corners
	,detach title
	,before upper = \tcbtitle\par\smallskip
	,coltitle = myp!85!black
	,fonttitle = \bfseries\sffamily
	,description font = \mdseries
	,separator sign none
	,segmentation style={solid, myp!85!black}
}
{th}
\tcbuselibrary{theorems,skins,hooks}
\newtcbtheorem[number within=chapter]{corollary}{Corollary}
{%
	enhanced
	,breakable
	,colback = myp!10
	,frame hidden
	,boxrule = 0sp
	,borderline west = {2pt}{0pt}{myp!85!black}
	,sharp corners
	,detach title
	,before upper = \tcbtitle\par\smallskip
	,coltitle = myp!85!black
	,fonttitle = \bfseries\sffamily
	,description font = \mdseries
	,separator sign none
	,segmentation style={solid, myp!85!black}
}
{th}

%================================
% LEMMA
%================================

\tcbuselibrary{theorems,skins,hooks}
\newtcbtheorem[number within=section]{lemma}{Lemma}
{%
	enhanced
	,breakable
	,colback = myg!10
	,frame hidden
	,boxrule = 0sp
	,borderline west = {2pt}{0pt}{myg}
	,sharp corners
	,detach title
	,before upper = \tcbtitle\par\smallskip
	,coltitle = myg!85!black
	,fonttitle = \bfseries\sffamily
	,description font = \mdseries
	,separator sign none
	,segmentation style={solid, myg!85!black}
}
{th}


\newtcbtheorem[number within=chapter]{Lemma}{Lemma}
{%
	enhanced
	,breakable
	,colback = myg!10
	,frame hidden
	,boxrule = 0sp
	,borderline west = {2pt}{0pt}{myg}
	,sharp corners
	,detach title
	,before upper = \tcbtitle\par\smallskip
	,coltitle = myg!85!black
	,fonttitle = \bfseries\sffamily
	,description font = \mdseries
	,separator sign none
	,segmentation style={solid, myg!85!black}
}
{th}

%================================
% CLAIM
%================================

\tcbuselibrary{theorems,skins,hooks}
\newtcbtheorem[number within=section]{claim}{Claim}
{%
	enhanced
	,breakable
	,colback = orange!10
	,frame hidden
	,boxrule = 0sp
	,borderline west = {2pt}{0pt}{orange}
	,sharp corners
	,detach title
	,before upper = \tcbtitle\par\smallskip
	,coltitle = orange!85!black
	,fonttitle = \bfseries\sffamily
	,description font = \mdseries
	,separator sign none
	,segmentation style={solid, orange!85!black}
}
{th}


\newtcbtheorem[number within=chapter]{Claim}{Claim}
{%
	enhanced
	,breakable
	,colback = orange!10
	,frame hidden
	,boxrule = 0sp
	,borderline west = {2pt}{0pt}{orange}
	,sharp corners
	,detach title
	,before upper = \tcbtitle\par\smallskip
	,coltitle = orange!85!black
	,fonttitle = \bfseries\sffamily
	,description font = \mdseries
	,separator sign none
	,segmentation style={solid, orange!85!black}
}
{th}

%================================
% EXAMPLE BOX
%================================

\newtcbtheorem[number within=section]{Example}{Example}
{%
	colback = myexamplebg
	,breakable
	,colframe = myexamplefr
	,coltitle = myexampleti
	,boxrule = 1pt
	,sharp corners
	,detach title
	,before upper=\tcbtitle\par\smallskip
	,fonttitle = \bfseries
	,description font = \mdseries
	,separator sign none
	,description delimiters parenthesis
}
{ex}

\newtcbtheorem[number within=chapter]{example}{Example}
{%
	colback = myexamplebg
	,breakable
	,colframe = myexamplefr
	,coltitle = myexampleti
	,boxrule = 1pt
	,sharp corners
	,detach title
	,before upper=\tcbtitle\par\smallskip
	,fonttitle = \bfseries
	,description font = \mdseries
	,separator sign none
	,description delimiters parenthesis
}
{ex}

%================================
% DEFINITION BOX
%================================

\newtcbtheorem[number within=section]{Definition}{Definition}{enhanced,
	before skip=2mm,after skip=2mm, colback=red!5,colframe=red!80!black,boxrule=0.5mm,
	attach boxed title to top left={xshift=1cm,yshift*=1mm-\tcboxedtitleheight}, varwidth boxed title*=-3cm,
	boxed title style={frame code={
			\path[fill=tcbcolback]
			([yshift=-1mm,xshift=-1mm]frame.north west)
			arc[start angle=0,end angle=180,radius=1mm]
			([yshift=-1mm,xshift=1mm]frame.north east)
			arc[start angle=180,end angle=0,radius=1mm];
			\path[left color=tcbcolback!60!black,right color=tcbcolback!60!black,
			middle color=tcbcolback!80!black]
			([xshift=-2mm]frame.north west) -- ([xshift=2mm]frame.north east)
			[rounded corners=1mm]-- ([xshift=1mm,yshift=-1mm]frame.north east)
			-- (frame.south east) -- (frame.south west)
			-- ([xshift=-1mm,yshift=-1mm]frame.north west)
			[sharp corners]-- cycle;
		},interior engine=empty,
	},
	fonttitle=\bfseries,
	title={#2},#1}{def}
\newtcbtheorem[number within=chapter]{definition}{Definition}{enhanced,
	before skip=2mm,after skip=2mm, colback=red!5,colframe=red!80!black,boxrule=0.5mm,
	attach boxed title to top left={xshift=1cm,yshift*=1mm-\tcboxedtitleheight}, varwidth boxed title*=-3cm,
	boxed title style={frame code={
			\path[fill=tcbcolback]
			([yshift=-1mm,xshift=-1mm]frame.north west)
			arc[start angle=0,end angle=180,radius=1mm]
			([yshift=-1mm,xshift=1mm]frame.north east)
			arc[start angle=180,end angle=0,radius=1mm];
			\path[left color=tcbcolback!60!black,right color=tcbcolback!60!black,
			middle color=tcbcolback!80!black]
			([xshift=-2mm]frame.north west) -- ([xshift=2mm]frame.north east)
			[rounded corners=1mm]-- ([xshift=1mm,yshift=-1mm]frame.north east)
			-- (frame.south east) -- (frame.south west)
			-- ([xshift=-1mm,yshift=-1mm]frame.north west)
			[sharp corners]-- cycle;
		},interior engine=empty,
	},
	fonttitle=\bfseries,
	title={#2},#1}{def}


%================================
% OPEN QUESTION BOX
%================================

\newtcbtheorem[number within=section]{open}{Open Question}{enhanced,
	before skip=2mm,after skip=2mm, colback=myp!5,colframe=myp!80!black,boxrule=0.5mm,
	attach boxed title to top left={xshift=1cm,yshift*=1mm-\tcboxedtitleheight}, varwidth boxed title*=-3cm,
	boxed title style={frame code={
			\path[fill=tcbcolback]
			([yshift=-1mm,xshift=-1mm]frame.north west)
			arc[start angle=0,end angle=180,radius=1mm]
			([yshift=-1mm,xshift=1mm]frame.north east)
			arc[start angle=180,end angle=0,radius=1mm];
			\path[left color=tcbcolback!60!black,right color=tcbcolback!60!black,
			middle color=tcbcolback!80!black]
			([xshift=-2mm]frame.north west) -- ([xshift=2mm]frame.north east)
			[rounded corners=1mm]-- ([xshift=1mm,yshift=-1mm]frame.north east)
			-- (frame.south east) -- (frame.south west)
			-- ([xshift=-1mm,yshift=-1mm]frame.north west)
			[sharp corners]-- cycle;
		},interior engine=empty,
	},
	fonttitle=\bfseries,
	title={#2},#1}{def}
\newtcbtheorem[number within=chapter]{Open}{Open Question}{enhanced,
	before skip=2mm,after skip=2mm, colback=myp!5,colframe=myp!80!black,boxrule=0.5mm,
	attach boxed title to top left={xshift=1cm,yshift*=1mm-\tcboxedtitleheight}, varwidth boxed title*=-3cm,
	boxed title style={frame code={
			\path[fill=tcbcolback]
			([yshift=-1mm,xshift=-1mm]frame.north west)
			arc[start angle=0,end angle=180,radius=1mm]
			([yshift=-1mm,xshift=1mm]frame.north east)
			arc[start angle=180,end angle=0,radius=1mm];
			\path[left color=tcbcolback!60!black,right color=tcbcolback!60!black,
			middle color=tcbcolback!80!black]
			([xshift=-2mm]frame.north west) -- ([xshift=2mm]frame.north east)
			[rounded corners=1mm]-- ([xshift=1mm,yshift=-1mm]frame.north east)
			-- (frame.south east) -- (frame.south west)
			-- ([xshift=-1mm,yshift=-1mm]frame.north west)
			[sharp corners]-- cycle;
		},interior engine=empty,
	},
	fonttitle=\bfseries,
	title={#2},#1}{def}



%================================
% EXERCISE BOX
%================================

\makeatletter
\newtcbtheorem{question}{Question}{enhanced,
	breakable,
	colback=white,
	colframe=myb!80!black,
	attach boxed title to top left={yshift*=-\tcboxedtitleheight},
	fonttitle=\bfseries,
	title={#2},
	boxed title size=title,
	boxed title style={%
		sharp corners,
		rounded corners=northwest,
		colback=tcbcolframe,
		boxrule=0pt,
	},
	underlay boxed title={%
		\path[fill=tcbcolframe] (title.south west)--(title.south east)
		to[out=0, in=180] ([xshift=5mm]title.east)--
		(title.center-|frame.east)
		[rounded corners=\kvtcb@arc] |-
		(frame.north) -| cycle;
	},
	#1
}{def}
\makeatother


%================================
% Question BOX
%================================

\makeatletter
\newtcbtheorem{qstion}{Question}{enhanced,
	breakable,
	colback=white,
	colframe=mygr,
	attach boxed title to top left={yshift*=-\tcboxedtitleheight},
	fonttitle=\bfseries,
	title={#2},
	boxed title size=title,
	boxed title style={%
		sharp corners,
		rounded corners=northwest,
		colback=tcbcolframe,
		boxrule=0pt,
	},
	underlay boxed title={%
		\path[fill=tcbcolframe] (title.south west)--(title.south east)
		to[out=0, in=180] ([xshift=5mm]title.east)--
		(title.center-|frame.east)
		[rounded corners=\kvtcb@arc] |-
		(frame.north) -| cycle;
	},
	#1
}{def}
\makeatother

\newtcbtheorem[number within=chapter]{wconc}{Wrong Concept}{
	breakable,
	enhanced,
	colback=white,
	colframe=myr,
	arc=0pt,
	outer arc=0pt,
	fonttitle=\bfseries\sffamily\large,
	colbacktitle=myr,
	attach boxed title to top left={},
	boxed title style={
		enhanced,
		skin=enhancedfirst jigsaw,
		arc=3pt,
		bottom=0pt,
		interior style={fill=myr}
	},
	#1
}{def}



%================================
% NOTE BOX
%================================

\usetikzlibrary{arrows,calc,shadows.blur}
\tcbuselibrary{skins}
\newtcolorbox{note}[1][]{%
	enhanced jigsaw,
	colback=gray!20!white,%
	colframe=gray!80!black,
	size=small,
	boxrule=1pt,
	title=\textbf{Note:},
	halign title=flush center,
	coltitle=black,
	breakable,
	drop shadow=black!50!white,
	attach boxed title to top left={xshift=1cm,yshift=-\tcboxedtitleheight/2,yshifttext=-\tcboxedtitleheight/2},
	minipage boxed title=1.5cm,
	boxed title style={%
		colback=white,
		size=fbox,
		boxrule=1pt,
		boxsep=2pt,
		underlay={%
			\coordinate (dotA) at ($(interior.west) + (-0.5pt,0)$);
			\coordinate (dotB) at ($(interior.east) + (0.5pt,0)$);
			\begin{scope}
				\clip (interior.north west) rectangle ([xshift=3ex]interior.east);
				\filldraw [white, blur shadow={shadow opacity=60, shadow yshift=-.75ex}, rounded corners=2pt] (interior.north west) rectangle (interior.south east);
			\end{scope}
			\begin{scope}[gray!80!black]
				\fill (dotA) circle (2pt);
				\fill (dotB) circle (2pt);
			\end{scope}
		},
	},
	#1,
}

%%%%%%%%%%%%%%%%%%%%%%%%%%%%%%
% SELF MADE COMMANDS
%%%%%%%%%%%%%%%%%%%%%%%%%%%%%%

\NewDocumentCommand{\EqM}{ m O{black} m}{%
	\tikz[remember picture, baseline, anchor=base] 
	\node[inner sep=0pt, outer sep=3pt, text=#2] (#1) {%
		\ensuremath{#3}%
	};    
}

\newcommand{\thm}[3][]{\begin{Theorem}{#2}{#1}#3\end{Theorem}}
\newcommand{\thmc}[3][]{\begin{theorem}{#2}{#1}#3\end{theorem}}
\newcommand{\cor}[3][]{\begin{corolary}{#2}{#1}#3\end{corolary}}
\newcommand{\corc}[3][]{\begin{corollary}{#2}{#1}#3\end{corollary}}
\newcommand{\lem}[3][]{\begin{lemma}{#2}{#1}#3\end{lemma}}
\newcommand{\clm}[3][]{\begin{claim}{#2}{#1}#3\end{claim}}
\newcommand{\wc}[3][]{\begin{wconc}{#2}{#1}\setlength{\parindent}{1cm}#3\end{wconc}}
\newcommand{\thmcon}[1]{\begin{Theoremcon}{#1}\end{Theoremcon}}
\newcommand{\ex}[3][]{\begin{Example}{#2}{#1}#3\end{Example}}
\newcommand{\exc}[3][]{\begin{example}{#2}{#1}#3\end{example}}
\newcommand{\dfn}[3][]{\begin{Definition}[colbacktitle=red!75!black]{#2}{#1}#3\end{Definition}}
\newcommand{\dfnc}[3][]{\begin{definition}[colbacktitle=red!75!black]{#2}{#1}#3\end{definition}}
\newcommand{\opn}[3][]{\begin{open}[colbacktitle=myp!75!black]{#2}{#1}#3\end{open}}
\newcommand{\opnc}[3][]{\begin{Open}[colbacktitle=myp!75!black]{#2}{#1}#3\end{Open}}
\newcommand{\qs}[3][]{\begin{question}{#2}{#1}#3\end{question}}
\newcommand{\pf}[2]{\begin{myproof}[#1]#2\end{myproof}}
\newcommand{\nt}[1]{\begin{note}#1\end{note}}
\newcommand*\circled[1]{\tikz[baseline=(char.base)]{
		\node[shape=circle,draw,inner sep=1pt] (char) {#1};}}
\newcommand\getcurrentref[1]{%
	\ifnumequal{\value{#1}}{0}
	{??}
	{\the\value{#1}}%
}
\newcommand{\getCurrentSectionNumber}{\getcurrentref{section}}
\newenvironment{myproof}[1][\proofname]{%
	\proof[\bfseries #1: ]%
}{\endproof}
\newcounter{mylabelcounter}
\makeatletter
\newcommand{\setword}[2]{%
	\phantomsection
	#1\def\@currentlabel{\unexpanded{#1}}\label{#2}%
}
\makeatother
\tikzset{
	symbol/.style={
		draw=none,
		every to/.append style={
			edge node={node [sloped, allow upside down, auto=false]{$#1$}}}
	}
}

%%%%%%%%%%%%%%%%%%%%%%%%%%%%%%%%%%%%%%%%%%%
% TABLE OF CONTENTS 
%%%%%%%%%%%%%%%%%%%%%%%%%%%%%%%%%%%%%%%%%%%

\usepackage{tikz}
\usepackage{titletoc}
\contentsmargin{0cm}
\titlecontents{chapter}[3.7pc]
{\addvspace{30pt}%
	\begin{tikzpicture}[remember picture, overlay]%
		\draw[fill=doc!90,draw=doc!90] (-7,-.1) rectangle (-0.7,.5);%
		\pgftext[left,x=-3.6cm,y=0.2cm]{\color{white}\Large\sc\bfseries Chapter\ \thecontentslabel};%
	\end{tikzpicture}\color{doc!90}\large\sc\bfseries}%
{}
{}
{\;\titlerule\;\large\sc\bfseries Page \thecontentspage
	\begin{tikzpicture}[remember picture, overlay]
		\draw[fill=doc!90,draw=doc!90] (2pt,0) rectangle (4,0.1pt);
\end{tikzpicture}}%
\titlecontents{section}[3.7pc]
{\addvspace{2pt}}
{\contentslabel[\thecontentslabel]{2pc}}
{}
{\hfill\small \thecontentspage}
[]
\titlecontents*{subsection}[3.7pc]
{\addvspace{-1pt}\small}
{}
{}
{\ --- \small\thecontentspage}
[ \textbullet\ ][]

\makeatletter
\renewcommand{\tableofcontents}{%
	\chapter*{%
		\vspace*{-20\p@}%
		\begin{tikzpicture}[remember picture, overlay]%
			\pgftext[right,x=15cm,y=0.2cm]{\color{doc!90}\Huge\sc\bfseries \contentsname};%
			\draw[fill=doc!90,draw=doc!90] (13,-.75) rectangle (20,1);%
			\clip (13,-.75) rectangle (20,1);
			\pgftext[right,x=15cm,y=0.2cm]{\color{white}\Huge\sc\bfseries \contentsname};%
	\end{tikzpicture}}%
	\@starttoc{toc}}
\makeatother



%%%%%%%%%%%%%%%%%%%%%%%%%%%%%%%%%%%%%%%%%%%
% Title Page 
%%%%%%%%%%%%%%%%%%%%%%%%%%%%%%%%%%%%%%%%%%%
\definecolor{doc}{RGB}{79, 134, 247}
\newcommand{\mytitleb}[4]{\begin{tikzpicture}[overlay,remember picture]
		
		% Background color
		\fill[
		black!2]
		(current page.south west) rectangle (current page.north east);
		
		% Rectangles
		\shade[
		left color=doc, 
		right color=doc!60,
		transform canvas ={rotate around ={45:($(current page.north west)+(0,-6)$)}}] 
		($(current page.north west)+(0,-6)$) rectangle ++(9,1.5);
		
		\shade[
		left color=lightgray,
		right color=lightgray!50,
		rounded corners=0.75cm,
		transform canvas ={rotate around ={45:($(current page.north west)+(.5,-10)$)}}]
		($(current page.north west)+(0.5,-10)$) rectangle ++(15,1.5);
		
		\shade[
		left color=lightgray,
		rounded corners=0.3cm,
		transform canvas ={rotate around ={45:($(current page.north west)+(.5,-10)$)}}] ($(current page.north west)+(1.5,-9.55)$) rectangle ++(7,.6);
		
		\shade[
		left color=doc,
		right color=doc!60,
		rounded corners=0.4cm,
		transform canvas ={rotate around ={45:($(current page.north)+(-1.5,-3)$)}}]
		($(current page.north)+(-1.5,-3)$) rectangle ++(9,0.8);
		
		\shade[
		left color=doc,
		right color=doc!60,
		rounded corners=0.9cm,
		transform canvas ={rotate around ={45:($(current page.north)+(-3,-8)$)}}] ($(current page.north)+(-3,-8)$) rectangle ++(15,1.8);
		
		\shade[
		left color=doc,
		right color=doc!60,
		rounded corners=0.9cm,
		transform canvas ={rotate around ={45:($(current page.north west)+(4,-15.5)$)}}]
		($(current page.north west)+(4,-15.5)$) rectangle ++(30,1.8);
		
		\shade[
		left color=doc,
		right color=doc!60,
		rounded corners=0.75cm,
		transform canvas ={rotate around ={45:($(current page.north west)+(13,-10)$)}}]
		($(current page.north west)+(13,-10)$) rectangle ++(15,1.5);
		
		\shade[
		left color=lightgray,
		rounded corners=0.3cm,
		transform canvas ={rotate around ={45:($(current page.north west)+(18,-8)$)}}]
		($(current page.north west)+(18,-8)$) rectangle ++(15,0.6);
		
		\shade[
		left color=lightgray,
		rounded corners=0.4cm,
		transform canvas ={rotate around ={45:($(current page.north west)+(19,-5.65)$)}}]
		($(current page.north west)+(19,-5.65)$) rectangle ++(15,0.8);
		
		\shade[
		left color=doc,
		right color=doc!60,
		rounded corners=0.6cm,
		transform canvas ={rotate around ={45:($(current page.north west)+(20,-9)$)}}] 
		($(current page.north west)+(20,-9)$) rectangle ++(14,1.2);
		
		% Year
		\draw[ultra thick,gray]
		($(current page.center)+(5,2)$) -- ++(0,-3cm) 
		node[
		midway,
		left=0.25cm,
		text width=5cm,
		align=right,
		black!75
		]
		{
			{\fontsize{25}{30} \selectfont \bf  Lecture\\[10pt] Notes}
		} 
		node[
		midway,
		right=0.25cm,
		text width=6cm,
		align=left,
		doc]
		{
			{\fontsize{72}{86.4} \selectfont #4}
		};
		
		% Title
		\node[align=center] at ($(current page.center)+(0,-5)$) 
		{
			{\fontsize{60}{72} \selectfont {{#1}}} \\[1cm]
			{\fontsize{16}{19.2} \selectfont \textcolor{doc}{ \bf #2}}\\[3pt]
			#3};
	\end{tikzpicture}
}


%%%%%%%%%%%%%%%%%%%
%  Todo Commands  %
%%%%%%%%%%%%%%%%%%%
\usepackage{xargs}
\usepackage[colorinlistoftodos]{todonotes}

\newcommandx{\unsure}[2][1=]{\todo[linecolor=red,backgroundcolor=red!25,bordercolor=red,#1]{#2}}
\newcommandx{\change}[2][1=]{\todo[linecolor=blue,backgroundcolor=blue!25,bordercolor=blue,#1]{#2}}
\newcommandx{\info}[2][1=]{\todo[linecolor=OliveGreen,backgroundcolor=OliveGreen!25,bordercolor=OliveGreen,#1]{#2}}
\newcommandx{\improvement}[2][1=]{\todo[linecolor=Plum,backgroundcolor=Plum!25,bordercolor=Plum,#1]{#2}}
\newcommandx{\thiswillnotshow}[2][1=]{\todo[disable,#1]{#2}}
\makeatletter

\usepackage{graphics}
\usepackage{titlesec}
\usepackage{xcolor} % If you haven't already included it for the color

% % Chapter style
% \titleformat{\chapter}[display]
%   {\normalfont\Large\raggedleft \color{doc}}
%   {\MakeUppercase{\chaptertitlename}%
%     \rlap{ \resizebox{!}{1.5cm}{\thechapter} \rule{5cm}{1.5cm}}}
%   {10pt}{\Huge}
% \titlespacing*{\chapter}{0pt}{30pt}{20pt}